\section{Proposed Algorithm}

For an $M \times N$ image, we denote $image(a)$ to denote the pixel value of pixel $a$. As we are working with
binary images, we suppose the value of foregroud pixel is 1 and that of background pixel is 0. All pixels in 
the edge of an image are considered to be background pixels. We have proposed two $2$-pass sequential algorithms
$RemSP$ and $ARemSP$ in this paper. These two algorithms are different in their first scan.

\subsection{\em{RemSP} Algorithm}
In the first scan of RemSP, we process image lines one by one using the forward scan mask as shown in figure \ref{fscan1}. We 
have used the decision tree proposed by \cite{Wu2009_LRPC} for determining the provisional label of current pixel
$e$ as we can reduce the number of neighbors using decision tree. Instead of examining all four neighbors of e, 
i.e., a,b,c and d, they proposed to examine the neighbors according to a desicion tree as shown in figure \ref{dtree}. Let $label$
denote the $2D$ array storing the labels and let $p$ denote equivalence array then according to $LRPC$ algorithm,
three functions used by this decision tree are defined as follows:

$1)$. The one-argument copy function, copy(a), contains one statement:
					$label(e) = p(label(a))$\\
$2)$. The two-argument copy function, copy(c,a), contains one statements:
				$label(e) = merge(p, label(c), label(a))$\\
$3)$. The new label function sets $count$ as $label(e)$, appends $count$ to array $p$, and increments $count$ by $1$.

However, the implementation of $merge$ operation in our proporsed algorithm $RemSP$ is different from that of in $LRPC$.
We have used the implementation of union-find proposed by $Rem$ \cite{Patwary2010_RemSP} for merge operation. $Rem$ 
integrates the Union operation with a compression technique known as Splicing $(sp)$. In the case when $rootx$ is to 
be moved to $p(rootx)$ it works as follows: just before this operation, $rootx$ is stored in a temporary variable $z$ and 
then, just before moving $rootx$ up to its parent $p(z)$, $p(rootx)$ is set to $p(rooty)$, making the subtree rooted at 
$rootx$ a sibling of $rooty$. This neither compromises the increasing parent property (because $p(rootx) < p(rooty)$) 
nor invalidates the set structures (because the two sets will have been merged when the operation ends.) The effect of $sp$
is that each new parent has a higher value than the value of the old parent, thus compressing the tree. The algorithm for
$merge$ is given as Algorithm \ref{alg:merge}. The full algorithm for $RemSP$ is
given as Algorithm \ref{alg:RemSP}.

\subsection{\em{ARemSP} Algorithm}
In the first scan of $ARemSP$, we process image lines two by two and processes pixels two by two using the mask shown in figure \ref{fscan2} suggested 
in \cite{He2012_ARun}. We will give the label to both $e$ and $g$ simultaneously. If both $e$ and $g$ are background pixels,
then needs to be done. If $e$ is a foreground pixel and there is no foreground pixel in the mask, we assign a new provisional label
to $e$ and if $g$ is a foreground pixel, we will give the label of $e$ to $g$. If there are foreground pixels in the mask, then we 
assign $e$ any label assigned to foreground pixels. In this case, if there is only one connected component in the mask then there is 
no need for label equivalance. Otherwise, if there are more than one connected component in the mask and as they are connected to $e$ so
all the labels of the connected components are equivalent labels and needs to be merged.For all the cases,one can refer \cite{He2012_ARun}.
In \cite{He2012_ARun}, they have used three arrays for merge operation. However, we have used the implementation of union-find proposed by 
$Rem$ \cite{Patwary2010_RemSP} for merge operation in $ARemSP$. The full
algorithm for $ARemSP$ is given as Algorithm \ref{alg:ARemSP}.





 
