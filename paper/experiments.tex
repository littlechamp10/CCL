\section{Experiments}
\label{sec:experiments}
For the experiments we used Hopper. Hopper is NERSC's first peta-flop system, a
Cray XE6, with a peak performance of $1.28$ Petaflops/sec, $153,216$ compute
cores for running scientific applications, $217$ Terabytes of memory, and $2$
Petabytes of online disk storage. All algorithms were implemented in C using OpenMP and compiled with gcc.

Our test dataset consists of $4$ types of image dataset: Texture, Arial,
Miscellaneous and NLCD. First three datasets are taken from the image database of the University of 
Southern California. The fourth dataset is taken from US National Cover Database
$2006$. All of the images are converted to binary images by means of MATLAB.
Texture,Arial and Miscellaneous dataset contain images of size $1024 \times
1024$ or less.
NCLD dataset contains images of size bigger than $3000 \times
4000$. The biggest image in the dataset is $22822 \times 20384$.

Firstly, we did the experiment over all the sequential algorithms. The result is given in Table \ref{table:seq}. 
We can see that {\em ARemSP} is best among all
the sequential algorithms. Then we tested the parallel algorithm {\em PARemSP} over all the images. 
Fig \ref{line}-\ref{linet} shows the speedup of the algorithm for 
NCLD image dataset. We get a maximum speedup of $20.1$ for image of size $22822 \times 20384$. 
Fig \ref{line} shows the speedup for {\em Phase-$I$} of \paremsp\ i.e. 
the local computation and fig \ref{linet} shows the overall speedup. We can see that there is not 
significant difference between both the speedups which shows that 
merge operation doesn't take much time. We have also shown the speedup for all
the other datasets in fig \ref{bar}. We get a maximum seedup of $10$ in this case as the images are small in size.  

