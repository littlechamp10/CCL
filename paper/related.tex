\subsection{Related Work}
\label{sec:relatedworks}

There exist lot of algorithms for computing Connected components in a given image.
As mentioned by one of the papers by Suzuki, these algorithms are categorized into mainly four groups:
1) methods by repeated passes over the data, (2) methods of two passes over the data,
(3) methods using hierarchical tree equivalent representations of the data, (4) parallel algorithms.
These all algorithms shares one common step, known as scanning step in which provisional label is given
to each of the pixel depending on its neighbors. In this paper, we concentrate on strategies to improve
the first two groups of methods.

There are two parts in scanning step: 1) examining neighbors of current pixel which already got label to
determine label for the current pixel. 2) storing label equivalence information to speed up the algorithm.
In the paper by Kesheng and etc , they have given two strategies to speedup the algorithm.
Firstly, they have implemented a decision tree by which the number of neighbors to be visited during the
first part of scanning step is  decreased in most of the cases. Secondly, they replaced the conventional
pointer based union-find algorithm used for storing label equivalence by array based union-find algorithm.
As they mentioned in their paper that array based approach will take less memory than the pointer based
approach so that will speedup the algorithm. They implemented union-find algorithm using
Link by Rank and Path Compression. They also compared their algorithm with the existing algorithms and
found that their algorithm performs better than the other existing algorithms.
We will call this algorithm as LRPC for our reference.

In an another paper by Suzuki and etc, they have implemented a different data structure to process label equivalence information.
They used three single dimensional arrays r[],next[] and tail[] in their implementation.
We will call this algorithm as Run for our reference.

In a paper Lifeng and etc, they gave an algorithm which process image lines two by two and process image pixels two by two.
They have used the data structure suggested in Run algorithm for processing label equivalence information.
They also compared their algorithm with the Run algorithm and found better results than that.
We will call this algorithm as Arun for our reference.

In a paper by Mostafa Ali Patwary and etc, they have compared all the different variations of union-find algorithms
over different graph data sets and found that RemSP, the implementation given by Rem is the best alomg all the
variations of union-find algorithm.

In this paper, we have given two different variations for CCL algorithm. In the first algorithm, we have used the
decision tree suggested by the LRPC algorithm for the first part of scanning step but for the second part
we have used RemSP instead of Link by Rank and Path Compression. We will call this RemSP for our reference.
In the second algorithm, we processed the image lines two by two as suggested by Arun algorithm but for the
second part, we used RemSP instead of data structure used by Arun algorithm. We will call this AremSP for our reference.

Then we have compared both of our veriations with LRPC, Run & Arun algorithms and found out that AremSP performs best
among all the algorithms.

Finally we have also given a parallel implementation of AremSP using OpenMP. Then we have compared with AremSP.
