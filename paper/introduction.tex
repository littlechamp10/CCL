\section{Introduction}

One of the most fundamental operations in pattern recognition is the labeling of connected components in a binary image.
Connected-component labeling is a procedure for assigning a unique label toeach object (or a connected component) in 
an image. Because these labels are key for other analytical procedures, connected-component labeling is an 
indispensable part of most applications in pattern recognition and computer vision, such as fingerprint identification,
character recognition, automated inspection, target recognition, face identification, medical image analysis, and 
computer-aided diagnosis. In many cases, it is also one of the most time-consuming tasks among other 
pattern-recognition algorithms \cite{Alnuweiri1992_Parallel}. Therefore, connected-component labeling continues 
to be an active area of research \cite{Gonzales_Digital,Agarwal2006_Efficient,Chang2004_Linear,Hayashi2001_Fast,
Hu2005_Fast,Knop1998_Parallel,Moga1997_Parallel,Wang2003_Parallel}.

In this paper, we only consider the problem of labeling binary images stored as $2D$ array of pixels. These images are 
typically the output from another image-processing step.A binary image contains two types of pixels: object pixel and 
background pixel.Generally, we consider value of object pixel as 1 and value of background pixel as 0. The connected-
component labeling problem is to assign a label to each object pixel so that connected object pixels have the same label.
In $2D$ images, there are two ways of defining connectedness: $4$-connectednss and $8$-connectedness. In the paper, we have 
only used the $8$-connectedness of the pixel.

There exist many algorithms for computing Connected components in a given image. These algorithms are categorized into
mainly four groups as mentioned by \cite{Suzuki2003_Linear}: (1) methods by repeated passes over the data, (2) methods 
of two passes over the data,(also called $2$-pass algorithm) (3) methods using hierarchical tree equivalent representations 
of the data, (4) parallel algorithms.

The group $1$ algorithms repeat passes through an image in forward and backward raster directions alternately to propagate 
the label equivalences until no labels change.

The group $2$ algorithms perform two passes: during the first pass, provisional labels are assigned to connected components;
the label equivalences are stored in a one-dimensional or a two-dimensional table array. After the first pass, the label 
equivalences are resolved by. This step is often performed by using a search algorithm such as the union-find algorithm.
The results of resolving are generally stored in a one-dimensional table. During the second pass, the provisional labels are 
replaced by the smallest equivalent label using the table. As the algorithm traverses image twice that's why these algorithms 
are called $2$-pass algorithms.

The group $3$-algortihms have been developed for the images represented by hierarchical tree structures i.e., n-ary tree such 
as bintree, quad-tree, octree, etc. The label equivalences are resolved by using a search algorithm such as the union-find algorithm.

The Parallel algorithms have been developed for parallel machine models such as a mesh connected massively parallel processor.

Hoverver all these algorithms shares one common step, known as scanning step in which provisional label is given to each of the
pixel depending on its neighbors.

In this paper,We have proposed a $2$-pass algorithm for labeling the connected components, based on union-find algorithm given by Rem.
We have also given the parallel implementation of proposed sequential algorithm. We also note that as our parallel algorithm is 
implemented in OpenMP it is therefore highly portable and does not rely on any low level hardware specific primitives.
