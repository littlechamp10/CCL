\section{Introduction}

One of the most fundamental operations in pattern recognition is the labeling of connected components in a binary image. 
Connected-component labeling is a procedure for assigning a unique label to each object (or a connected component) in an 
image. Because these labels are key for other analytical procedures, connected-component labeling is an indispensable part 
of most applications in pattern recognition and computer vision, such as fingerprint identification, character recognition,
automated inspection, target recognition, face identification, medical image analysis, and computer-aided diagnosis. 
In many cases, it is also one of the most time-consuming tasks among other pattern-recognition algorithms 
\cite{Alnuweiri1992_Parallel}.Therefore, connected-component labeling continues to be an active area of research 
\cite{Gonzales_Digital,Agarwal2006_Efficient,Chang2004_Linear,Hayashi2001_Fast,Hu2005_Fast,Knop1998_Parallel,
Moga1997_Parallel,Wang2003_Parallel}.
