\section{Related Work}
\label{sec:relatedworks}

In any $2$-pass algorithm,there are two parts in scanning step: $1)$ examining neighbors of current pixel which already
got label to determine label for the current pixel. $2)$ storing label equivalence information to speed up the algorithm. 
In \cite{Wu2009_LRPC},they have given two strategies to speedup the algorithm. First strategy reduces the average number
of neighbors accessed by factor of $2$ through the use of decision tree. Second strategy replaces the conventional pointer
based union-find algorithm used for storing label equivalence by array based union-find algorithm. They proves that array 
based approach will take less memory than the pointer based approach so that will speedup the algorithm. They implemented
union-find algorithm using Link by Rank and Path Compression. We will call this algorithm as LRPC for our reference.

In \cite{He2008_Run}, they replaced the union-find data structure by a different data structure to process label 
equivalence information.In their algorithm, at any point, all provisional labels that are assigned to a connected 
component found so far during the first scan are combined in a set $S(r)$, where $r$ is the smallest label and is 
referred to as the representative label.They used $rtable[]$ for storing representative label of a set, $next[]$ to 
find the next element in the set and $tail[]$ to find the last element of the set.We have implemented an algorithm 
with decision tree along with the above data structure. We will call it as Run for our reference.

In \cite{He2012_ARun}, they replaced the first part of scanning step by a new algorithm which process image lines two
by two and process image pixels two by two. They used the same data structure given in \cite{He2008_Run} for 
processing label equivalence information.They also proved that as the number of lines to be scanned will be halved so 
it will speedup the algorithm. We will call this algorithm as Arun for our reference.

In \cite{Patwary2010_RemSP}, they compared all the different variations of union-find algorithms over different graph 
data sets and found that RemSP, the implementation given by Rem is the best alomg all the variations of union-find 
algorithm.

In this paper, we have given two different variations for CCL algorithm. In the first variation, we have used the 
decision tree suggested by the LRPC algorithm for the first part of scanning step but for the second part we have used
RemSP instead of Link by Rank and Path Compression. We will call this RemSP for our reference.

In the second one, we processed the image lines two by two as suggested by \cite{He2012_ARun} but for the second part,
we used RemSP instead of data structure used by \cite{He2012_ARun}. We will call this ARemSP for our reference.Then we
have compared both of our veriations with LRPC, Run and Arun algorithms and
found out that ARemSP performs best among all the algorithms.

Finally we have also given a parallel implementation of ARemSP using OpenMP.We used the parallel implementation of 
RemSP given in \cite{Patwary2012_PARemSP} Then we have compared it with ARemSP.
