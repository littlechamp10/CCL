\vspace{\sectionSpace}
\section{Introduction}

One of the most fundamental operations in pattern recognition is the labeling of connected components in a binary image.
Connected component labeling (CCL) is a procedure for assigning a unique label
to each object (or a connected component) in an image. Because these labels are key for other analytical procedures, connected component labeling is an 
indispensable part of most applications in pattern recognition and computer vision, such as fingerprint identification,
character recognition, automated inspection, target recognition, face identification, medical image analysis, and 
computer-aided diagnosis. In many cases, it is also one of the most time-consuming tasks among other 
pattern-recognition algorithms\cite{Alnuweiri1992_Parallel}. Therefore, connected component labeling continues 
to be an active area of research \cite{Gonzales_Digital,Agarwal2006_Efficient,Chang2004_Linear,Hayashi2001_Fast,
Hu2005_Fast,Knop1998_Parallel,Moga1997_Parallel,Wang2003_Parallel}.

%%%%%%%%%%%%%THIS PARAGRAPH IS DOESN'T FLOW - Consider revising it or moving it%%%%%%%%%%%%%%%%%%%%%%%%%%%%%%%%%
%In this paper, we only consider the problem of labeling binary images stored as $2D$ array of pixels. These images are 
%typically the output from another image-processing step.A binary image contains two types of pixels: object pixel and 
%background pixel.Generally, we consider value of object pixel as 1 and value of background pixel as 0. The connected-
%component labeling problem is to assign a label to each object pixel so that connected object pixels have the same label.
%In $2D$ images, there are two ways of defining connectedness: $4$-connectednss and $8$-connectedness. In the paper, we have 
%only used the $8$-connectedness of the pixel.

There exist many algorithms for computing connected components in a given image. These algorithms are categorized into
mainly four groups %as mentioned by%
\cite{Suzuki2003_Linear}
: $1)$ repeated pass
algorithms\cite{Haralick1981_Repeated,Hashizume1990_Algorithm}, $2)$ two-pass
algorithms\cite{gotoh1990high,gotoh1987component,komeichi1989video,lumia1983new,lumia1983new_c,naoi1995high,rosenfeld1970connectivity,
rosenfeld1966sequential,shirai1987labeling}
$3)$ Algorithms with hierarchical tree equivalent representations of the
data\cite{dillencourt1992general,gargantini1982separation,hecquard1991connected,samet1981connected,samet1985computing,samet1988efficient,
samet1986improved,tamminen1984efficient},
$4)$ parallel
algorithms\cite{bhattacharya1996connected,choudhary1994connected,
hirschberg1979computing,manohar1989connected,nassimi1980finding,olariu1993fast}.
The repeated pass algorithms perform repeated passes over an image in forward and backward raster directions alternately
to propagate the label equivalences until no labels change.
In two-pass algorithms, during the first pass, provisional labels are assigned to connected components;
the label equivalences are stored in a one-dimensional or a two-dimensional table array. After the first pass, the label 
equivalences are resolved by some search. This step is often performed by using
a search algorithm such as the union-find algorithm.
The results of resolving are generally stored in a one-dimensional table. During the second pass, the provisional labels are 
replaced by the smallest equivalent label using the table. As the algorithm traverses image twice that's why these algorithms 
are called two-pass algorithms.
In algorithms that employ hierarchical tree structures i.e., n-ary tree such as binary-tree, quad-tree, octree, etc., the
label equivalences are resolved by using a search algorithm such as the union-find algorithm.
Lastly, the parallel algorithms have been developed for parallel machine models such as a mesh connected massively parallel processor.
However all these algorithms share one common step, known as scanning step in
which provisional label is given to each of the pixel depending on its neighbors.

In this paper we focus on two-pass CCL algorithms. \cite{Wu2009_LRPC}, and \cite{He2012_ARun} are two
developed techniques for two-pass CCL algorithms.
The algorithm in \cite{Wu2009_LRPC}, which we refer to as \lrpc, uses a decision tree to assign provisional labels and an 
array-based union-find data structure
to store label equivalence information. However, the technique employed for
union-find, Link by Rank and Path Compression is not the best technique available \cite{Patwary2012_PARemSP}. 
The algorithm in \cite{He2012_ARun}, which we refer to as \arun, employs a special scan order over the data and three linear
arrays instead of the conventional union-find data structure. There
exists a parallel implementation of \arun\ on TILE\-64 many core
platform\cite{Chen2013_PARun}. According to the
experimental results given in \cite{Chen2013_PARun}, the parallel implementation is able to achieve a speedup of 10 on
32 processor units. 
As the parallel implementation is hardware specific and
parallel efficiency is less than 33\% , thus this unconventional
implementation is not suited for parallel implementation.

\begin{table}[h]
\caption{Abbreviations used in the paper and their brief description }
\centering
\begin{tabular}{l p{5cm}} 
\hline\hline
Abbreviation & Description\\ [0.5ex] 
\hline 
CCL & Connected Component Labeling\\[1ex]
\arun\ & CCL algorithm suggested by \cite{He2012_ARun}\\[1ex]
\remsp\ & union-find technique proposed by {\em Rem}
\cite{Patwary2010_RemSP}\\[1ex]
\aremsp\ & CCL algorithm proposed in our paper using scan strategy of \arun\ and
\remsp\\[1ex]
\paremsp\ & Parallel implementation of \aremsp\ proposed in our paper\\[1ex]
\lrpc\ & CCL algorithm suggested by \cite{Wu2009_LRPC}\\[1ex]
 \nremsp\ & CCL algorithm proposed in our paper using scan strategy of \lrpc\ 
and \remsp\\[1ex]
\hline
\end{tabular}
\label{table:abr} 
\end{table}

We propose two two-pass algorithms for labeling the connected components that we
call as \aremsp\ and \nremsp, which are based on \rems\ union-find algorithm
\remsp\ \cite{Patwary2010_RemSP, Dijkstra1976_RemSP} and the scan strategy of \arun\ and
\lrpc\ algorithms.
Since \rems\ union-find is an interleaved algorithm which implements immediate parent check test and 
compression technique called {\em Splicing} \cite{Patwary2010_RemSP,
Dijkstra1976_RemSP}, our proposed sequential two-pass algorithm \aremsp\ is $39$\% faster than \lrpc\
and $4$\% faster than \arun.
Another advantage of using \rems\ union-find approach is that its parallel implementation is shown to scale better
with increasing number of processor \cite{Patwary2012_PARemSP}. Parallel \rems\ union-find implementation thus allows us to
process the pixels of the image in any order. Therefore, we propose a parallel implementation of our proposed
sequential two-pass CCL algorithm \aremsp\ which we call as \paremsp. For
scalability, our algorithm in the first pass, divides the image into equal proportions and executes the scan strategy of \arun\ algorithm along with \remsp\
concurrently on each portion of the image. To merge the provisional labels on the image boundary, we use the parallel version
of \remsp \cite{Patwary2012_PARemSP}. Our experiments show
the scalability of \paremsp\ achieving speedups up to $20.1$ using $24$ cores
on shared memory architecture for an image of size $22,822 \times 20,384$.
Additionally, the parallel algorithm does not make use of any hardware specific routines, and thus is highly portable.

The remainder of this paper is organized as follows. In section
\ref{sec:related_works}, we provided related work on connected component labeling.
In section \ref{sec:proposed_algorithm}, we propose our sequential two-pass CCL
algorithms \nremsp\ and \aremsp\ and parallel version of \aremsp\ in section
\ref{sec:parallel_algo}.
We present our experimental methodology and results in section
\ref{sec:experiments}. We conclude our work in section
\ref{conclusion}. The abbreviations used in the paper and their brief
description is given in Table \ref{table:abr}.


