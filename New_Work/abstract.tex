\begin{abstract}
Connected Component Labeling (CCL) is one of the most important step in pattern
recognition and image processing. It assigns labels to the pixels such that
adjacent pixels sharing the same features are assigned the same label.
Typically, CCL requires several passes over the data. For example, in a two-pass technique, 
each pixel is given a provisional label in the first pass whereas an actual
label is assigned in the second pass. Suzuki et al have proposed two
algorithms for CCL with two-pass technique, which we refer to as \lrpc\ and
\arun\cite{Wu2009_LRPC, He2012_ARun}.\ The \lrpc\ algorithm uses a decision tree
to assign provisional labels and an array-based union-find data structure to store label equivalence information.
The \arun\ algorithm employs a special scan order over the data and three
linear arrays instead of the conventional union-find data structure.
As the resolution of images becomes higher due to better digital camera
technology and dynamic image matching has been common for surveillance, the
connected component problem becomes a data-intensive application and hence the need for exploiting parallelism arises.
The \arun\ algorithm has been parallelized for Tile64 many-core platform,
however the achieved speedup of 10 on 32 cores is sub-linear.

We present a scalable parallel two-pass CCL algorithm, called \paremsp,\ 
which employs scan strategy of \arun\ algorithm and the best union-find
technique called \remsp,\ which uses \rems\ algorithm for storing label
equivalence information of pixels in a 2-D image. In the first pass, we divide the image among threads and each
thread runs the scan strategy of \arun\ algorithm along with \remsp\
simultaneously. As \remsp\ is easily parallelizable, we use the parallel
version of \remsp\ for merging the pixels on the boundary. Our experiments
show the scalability of \paremsp\ achieving speedups up to $20.1$ using $24$
cores on shared memory architecture using {\em OpenMP} for an image of size
$22,822 \times 20,384$. We find that our proposed parallel algorithm achieves
linear scaling for a  large resolution fixed problem size while the number of
processing elements are increased. Additionally, the parallel algorithm does
not make use of any hardware specific routines, and thus is highly portable.
\end{abstract}
