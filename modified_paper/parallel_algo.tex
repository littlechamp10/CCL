\section{Parallelizing \aremsp\ Algorithm}
\label{sec:parallel_algo}

In the following we describe how one can run \aremsp\ algorithm in parallel
on a shared memory system. In doing so we make the assumption about memory
model as stated in {\em OpenMP} when we use atomic directive. We assume that
memory read/write operations are atomic and any operations issued concurrently 
by different processors will be executed in some unknown sequential order if
no ordering constructs are being used. However, two dependent operations issued
by the same processor will always be applied in the same order as they are
issued.\cite{Patwary2012_PARemSP}

In \paremsp, we divide the image among threads row-wise. The image is
divided into chunks of equal size and given to the threads. In the first step,
each thread run Phase-I of \aremsp\ on it's chunk simultaneously. We
initialize the label to the start index of the thread for every thread so that no 
two pixels in the image have the same label after the first step. After the first step, 
each pixel is given a provisional label. Now the pixels at the boundary of each chunk need to 
be merged to get the final labels. In the second step, we merge the boundary pixels using parallel 
implementation of Rem's Algorithm \cite{Patwary2012_PARemSP}. We implement the parallel algorithm using
OpenMP directives {\em pragma omp parallel} and {\em pragma omp for}. The
pseudo code for parallel implementation of Rem's Algorithm is given as Algorithm
\ref{alg:merger}. The pseudo code of \paremsp\ is given as Algorithm
\ref{alg:PARemSP}.
\begin{algorithm}[H]
\small
{
	\caption{Pseudo-code for PARemSP}
	\label{alg:PARemSP}
	\textbf{Input:} $2D$ array $image$ containing the pixel values \\
	\textbf{Output:} $2D$ array $label$ containing the final labels
	\begin{algorithmic}[1]
	\Function{PARemSP}{$image$}
		\State $num iter \gets row/2 $ \Comment{As we are processing $2$ rows at a time}
		\State \# pragma omp parallel
			\State $chunk \gets numiter/number of threads $ 
			\State $size \gets 2 \times chunk$
			\State $start \gets $ start index of the thread
			\State $count \gets start \times col$
			\State \# pragma omp for
				\State $ARemSP-I(image)$
			\State \# pragma omp for
				\For {$i=size$ to $row-1$}
					  \For{$col$ in $row$}
					  	\If{$label(e) \neq 0$}
					  		\If{$label(b) \neq 0$}
					  			\State $merger(p,label(e),label(b))$
					  		\Else
					  			\If{$label(a) \neq 0$}
									\State $merger(p,label(e),label(a))$
								\EndIf
								\If{$label(c) \neq 0$}
									\State $merger(p,label(e),label(c))$
								\EndIf
							\EndIf
						\EndIf
					\EndFor
					\State $i \gets i + size$
				\EndFor		
		\State $flatten(p,count)$ 
		\For{$row$ in $image$}  
			\For{$col$ in $row$}
				\State $label(e) \gets p[label(e)]$
			\EndFor
		\EndFor		
	\EndFunction
	\end{algorithmic}	
}
\end{algorithm}


 \begin{algorithm}[ht]
\small
{
	\caption{Pseudo-code for merger}
	\label{alg:merger}
	\textbf{Input:} $1D$ array $p$ and two nodes $x$ and $y$ \\
	\textbf{Output:} The root of united tree 
	\begin{algorithmic}[1]
	\Function{merger}{$p$,$x$,$y$} 
		\State $rootx \gets x, rooty \gets y$
		\While { $p[rootx] \neq p[rooty]$ }
			\If {$p[rootx] > p[rooty]$}
				\If{$rootx = p[rootx]$}
					\State $omp\_set\_lock(\&(lock\_array[rootx]))$
					\State $success \gets 0$
					\If{$rootx = p[rootx]$}
						\State $p[rootx] \gets p[rooty]$
						\State $success \gets 1$
					\EndIf
					\State $omp\_unset\_lock(\&(lock\_array[rootx]))$
					\If{$success =1$}  
						\State break
					\EndIf
				\EndIf
				\State $z \gets p[rootx], p[rootx] \gets p[rooty], rootx \gets z$
			\Else
				\If{$rooty = p[rooty]$}
					\State $omp\_set\_lock(\&(lock\_array[rooty]))$
					\State $success \gets 0$
					\If{$root = p[rooty]$}
						\State $p[rooty] \gets p[rootx]$
						\State $success \gets 1$
					\EndIf
					\State $omp\_unset\_lock(\&(lock\_array[rooty]))$
					\If{$success =1$}  
						\State break
					\EndIf
				\EndIf
				\State $z \gets p[rooty], p[rooty] \gets p[rootx], rooty \gets z$
			\EndIf
		\EndWhile
		\State \Return{$p[rootx]$}
	\EndFunction
	\end{algorithmic}	
}
\end{algorithm}
