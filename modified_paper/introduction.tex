\section{Introduction}

One of the most fundamental operations in pattern recognition is the labeling of connected components in a binary image.
Connected-component labeling(CCL) is a procedure for assigning a unique label to each object (or a connected component) in 
an image. Because these labels are key for other analytical procedures, connected-component labeling is an 
indispensable part of most applications in pattern recognition and computer vision, such as fingerprint identification,
character recognition, automated inspection, target recognition, face identification, medical image analysis, and 
computer-aided diagnosis. In many cases, it is also one of the most time-consuming tasks among other 
pattern-recognition algorithms \cite{Alnuweiri1992_Parallel}. Therefore, connected-component labeling continues 
to be an active area of research \cite{Gonzales_Digital,Agarwal2006_Efficient,Chang2004_Linear,Hayashi2001_Fast,
Hu2005_Fast,Knop1998_Parallel,Moga1997_Parallel,Wang2003_Parallel}.

%%%%%%%%%%%%%THIS PARAGRAPH IS DOESN'T FLOW - Consider revising it or moving it%%%%%%%%%%%%%%%%%%%%%%%%%%%%%%%%%
%In this paper, we only consider the problem of labeling binary images stored as $2D$ array of pixels. These images are 
%typically the output from another image-processing step.A binary image contains two types of pixels: object pixel and 
%background pixel.Generally, we consider value of object pixel as 1 and value of background pixel as 0. The connected-
%component labeling problem is to assign a label to each object pixel so that connected object pixels have the same label.
%In $2D$ images, there are two ways of defining connectedness: $4$-connectednss and $8$-connectedness. In the paper, we have 
%only used the $8$-connectedness of the pixel.

There exist many algorithms for computing connected components in a given image. These algorithms are categorized into
mainly four groups %as mentioned by%
\cite{Suzuki2003_Linear}
: $1)$ repeated pass algorithms, $2)$ two-pass algorithms $3)$ Algorithms with hierarchical tree equivalent representations 
of the data, $4)$ parallel algorithms.
The repeated pass algorithms perform repeated passes over an image in forward and backward raster directions alternately
to propagate the label equivalences until no labels change.
In two-pass algorithms, during the first pass, provisional labels are assigned to connected components;
the label equivalences are stored in a one-dimensional or a two-dimensional table array. After the first pass, the label 
equivalences are resolved by some search. This step is often performed by using
a search algorithm such as the union-find algorithm.
The results of resolving are generally stored in a one-dimensional table. During the second pass, the provisional labels are 
replaced by the smallest equivalent label using the table. As the algorithm traverses image twice that's why these algorithms 
are called two-pass algorithms.
In algorithms that employ hierarchical tree structures i.e., n-ary tree such as binary-tree, quad-tree, octree, etc., the
label equivalences are resolved by using a search algorithm such as the union-find algorithm.
Lastly, the parallel algorithms have been developed for parallel machine models such as a mesh connected massively parallel processor.
Hoverver all these algorithms shares one common step, known as scanning step in which provisional label is given to each of the
pixel depending on its neighbors.

In this paper we focus on two-pass CCL algorithms. Link by Rank and Path
Compression(\lrpc) \cite{Wu2009_LRPC}, and \arun \cite{He2012_ARun} are two
developed techniques for two-pass CCL algorithms.
The \lrpc\ algorithm uses a decision tree to assign provisional labels and an array-based union-find datastructure
to store label equivalence information. The \arun\ algorithm employs a special scan order over the data and three linear
arrays instead of the conventional union-find datastructure. 
We propose a two-pass algorithm for labeling the connected components called
\aremsp, which is based on \rems\ union-find algorithm \cite{Patwary2010_RemSP}
and the scan strategy of \arun\ algorithm. Since \rems\ union-find is an
interleaved algorithm which implements immediate parent check test and 
compression technique called {\em Splicing} \cite{Patwary2010_RemSP}, our
proposed sequential two-pass algorithm \aremsp\ is $39$\% faster than \lrpc\
and $4$\% faster than \arun.
Another advantage of using \rems\ union-find approach is that its parallel implementation is shown to scale better
with increasing number of processor \cite{Patwary2012_PARemSP}. Parallel \rems\ union-find implementation thus allows us to
process the pixels of the image in any order. Therefore, we propose a parallel implementation of our proposed
sequential algorithm two-pass CCL algorithm called \paremsp. For scalability,
our algorithm in in the first pass, divides the image among threads and each
thread runs the scan strategy of \arun\ algorithm along with \remsp\
simultaneously. To merge the provisional labels on the image boundary, we use the parallel version of \remsp \cite{Patwary2012_PARemSP}. Our experiments show
the scalability of \paremsp\ achieving speedups up to $20.1$ using $24$ cores
on shared memory architecture for an image of size $22,822 \times 20,384$.
Additionally, the parallel algorithm does not make use of any hardware specific routines, and thus is highly portable.

The remainder of this paper is organized as follows. In section
\ref{sec:related_works}, we provided related work on connected component labeling.
In section \ref{sec:proposed_algorithm}, we propose our sequential two-pass CCL
algorithm \aremsp\ and it's parallel version in section
\ref{sec:parallel_algo}.
We present our experimental methodology and results in section
\ref{sec:experiments}. We conclude our work and propose future work in section \ref{}.


