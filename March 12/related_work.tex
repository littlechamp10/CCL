\vspace{\sectionSpace}
\section{Related Work}
\label{sec:related_works}

As mentioned in \cite{Suzuki2003_Linear}, there exist different types of CCL
algorithms. Repeated pass or multi pass algorithm repeatedly scans the image
forward and backward alternatively to give labels until no further changes can
be made to the assigned
pixels\cite{Haralick1981_Repeated,Hashizume1990_Algorithm}. 
The algorithm in \cite{Suzuki2003_Linear}, which we call as {\em Suzuki's} algorithm modifies the
conventional multi pass algorithm using one-dimensional table. There exists a
parallel implementation of {\em Suzuki's} algorithm using OpenMP in
\cite{Niknam2010_Parallel}. According to experimental results in
\cite{Niknam2010_Parallel}, the parallel implementation gets maximum speedup of
2.5 on 4 threads.

In any two-pass algorithm, there are two steps in scanning step: $1)$ examining neighbors of current pixel which already
assigned labels to determine label for the current pixel, $2)$ storing label equivalence information to speed up the algorithm. 

The algorithm in \cite{Wu2009_LRPC}, which we refer to as \lrpc, provides two strategies to improve the running time of the algorithm.
First strategy employs a decision tree, which reduces the average number of neighbors accessed by a factor of two.
Second strategy replaces the conventional pointer based union-find algorithm, which is used for storing label equivalence,
by adopting array based union-find algorithm that uses less memory. The
union-find algorithm is implemented using Link by Rank and Path Compression technique. %For our reference, we call this algorithm \lrpc.

The union-find data structure in \cite{He2008_Run} is replaced by a different data structure to process label equivalence
information. In this algorithm, at any point, all provisional labels that are assigned to a connected 
component found thus far during the first scan are combined in a set $S(r)$, where $r$ is the smallest label and is 
referred to as the representative label. The algorithm employs $rtable$ for storing representative label of a set, $next$ to 
find the next element in the set and $tail$ to find the last element of the set.

%We have implemented an algorithm with decision tree along with the above data structure. We will call it as Run for our reference.

In another strategy, which we call \arun, the first part of scanning step employs a scanning technique, which processes image two lines 
at a time and process two image pixels at a time \cite{He2012_ARun}. This
algorithm uses the same data structure given in \cite{He2008_Run} for processing label equivalence information. The scanning technique reduces the number lines to be
processed by half thereby improving the speed of the two-pass CCL method. %For our reference, we call this algorithm \arun.

In this paper, we provide two different implementations of two-pass CCL algorithm. These two algorithms are different in their
first scan step. In the first implementation called \nremsp, we have used the 
decision tree suggested by the \lrpc\ algorithm for the first part of scanning step but for the second part we have used
\rems\ union-find approach instead of Link by Rank and Path Compression technique.
\cite{Patwary2010_RemSP} compares all of the different variations of union-find algorithms over different graph data sets and found that
\rems\ implementation is best among all the variations. Thus in our second implementation, called \aremsp, we process the image lines
two by two as suggested by \cite{He2012_ARun} but for the second step
we use \remsp\ instead of the data structure used by \cite{He2012_ARun}.

We have compared both of our proposed implementations with \lrpc, \run, and
\arun\ algorithms and find that \aremsp\ performs best among all the algorithms.
Finally we have also provided a shared memory parallel implementation of
\aremsp\ called \paremsp\ using OpenMP.
We use the parallel implementation of \remsp\ given in \cite{Patwary2012_PARemSP}. 
